%-=-=-=-=-=-=-=-=-=-=-=-=-=-=-=-=-=-=-=-=-=-=-=-=
%	DOCUMENT CLASS & PACKAGES
%-=-=-=-=-=-=-=-=-=-=-=-=-=-=-=-=-=-=-=-=-=-=-=-=

\documentclass[a4paper, 11pt]{article}
\usepackage{markolsonworksheet}
\usepackage{markolsoncolorsthlm}
\usepackage{markolsonmath}


%-=-=-=-=-=-=-=-=-=-=-=-=-=-=-=-=-=-=-=-=-=-=-=-=
%	BEGIN DOCUMENT
%-=-=-=-=-=-=-=-=-=-=-=-=-=-=-=-=-=-=-=-=-=-=-=-=

\begin{document}

\maketitle % Print the title section

%-=-=-=-=-=-=-=-=-=-=-=-=-=-=-=-=-=-=-=-=-=-=-=-=
%	SAGE SILENT 
%-=-=-=-=-=-=-=-=-=-=-=-=-=-=-=-=-=-=-=-=-=-=-=-=


%-=-=-=-=-=-=-=-=-=-=-=-=-=-=-=-=-=-=-=-=-=-=-=-=
%	WORKSHEET INSTRUCTIONS
%-=-=-=-=-=-=-=-=-=-=-=-=-=-=-=-=-=-=-=-=-=-=-=-=
\centering{Linear Regression Olympic Gold Medal Swimming Times for 100m Freestyle}\\
\vspace{0.5cm}

\begin{multicols}{2}
\begin{tabular}{ | l | l | c |}
    \hline
    Year    &     Men Gold Medalist         & Time \\
    \hline
    1912    &    D. Kahanamoku, USA     &    63.4 \\
    1920    &    D. Kahanamoku, USA     &    61.4 \\
    1924    &    J. Weissmuller, USA    &    59.0 \\
    1928    &    J. Weissmuller, USA    &    58.6 \\
    1932    &    Y. Miyazaki, Japan     &    58.2 \\
    1936    &    F. Csik, Hungary       &    57.6 \\
    1948    &    R. Ris, USA            &    57.3 \\
    1952    &    C. Scholes, USA        &    57.4 \\
    1956    &    J. Henricks, Austalia  &    55.4 \\
    1960    &    J. Devitt, Australia   &    55.2 \\
    1964    &    D. Schollander, USA    &    53.4 \\
    1968    &    M. Wenden, Australia   &    52.2 \\
    1972    &    M. Spitz, USA          &    51.22 \\
    1976    &    J. Montgomery, USA     &    49.99 \\
    1980    &    J. Woithe, E. Germany  &    50.40 \\
    1984    &    R. Gains, USA          &    49.80 \\
    1988    &    M. Biondi, USA         &    48.63 \\
    1992    &    A. Popov, Russia       &    49.02 \\
    1996    &    A. Popov, Russia       &    48.74 \\
    2000    &    P. vd Hoog., Ned       &    48.30  \\
    2004    &    P. vd Hoog., Ned       &    48.17  \\
    2008    &    A. Bernard, Fra        &    47.21 \\
		2012		&    N. Adrian, USA         & 	47.52 \\
			%2016		&		 K. Chalmers, AUS				&			47.58\\
    \hline
    \end{tabular}

\columnbreak
\begin{tabular}{ | l | l | l |}
    \hline
    Year    &     Women Gold Medalist         & Time\\
    \hline
    1912    &    F. Durack, Australia          &    82.2 \\
    1920    &    E. Bleibtry, USA               &    73.6 \\
    1924    &    E. Lackie, USA                 &    72.4 \\
    1928    &    A. Osipowich, USA              &    71.0 \\
    1932    &    H. Madison, USA                &    66.8 \\
    1936    &    H. Mastenbroek, Holland        &    65.9 \\
    1948    &    G. Andersen, Denmark           &    66.3 \\
    1952    &    K. Szoke, Hungary              &    66.8 \\
    1956    &    D. Fraser, Australia           &    62.0 \\
    1960    &    D. Fraser, Australia           &    61.2 \\
    1964    &    D. Fraser, Australia           &    59.5 \\
    1968    &    J. Henne, USA                  &    60.0 \\
    1972    &    S. Nielson,USA                 &    58.59 \\
    1976    &    K. Ender, E. Germany           &    55.65 \\
    1980    &    B. Krause, E. Germany          &    54.79 \\
    1984    &    C. Steinseife, USA (tie)       &    55.92 \\
    1984    &    N. Hogshead, USA (tie)         &    55.92 \\
    1988    &    K. Otta, E. Germany            &    54.93 \\
    1992    &    Z. Young, China                &    54.64 \\
    1996    &    L. Jingyi, China               &    54.50 \\
		2000    &    I. de Bruijn, Netherlands      &    53.83 \\
		2004		& 	 J. Henry, Australia            &   53.84 \\
		2008 		& 	 B. Steffen, Gemany		        	&   53.12 \\
		2012		&	 R. Kromowidjojo, Netherlands			&   53.00 \\
		%2016		&				P. Oleksiak									& 	52.70 \\
    \hline
    \end{tabular}
\end{multicols}
\begin{enumerate}
\item  Make a scatter plot for each set of data (plot the graph set).
\item  Find the equation of the line that best fits the given data.
\item  Graph the regression line (plot the line of best fit).
\item  Does your linear equation predict the following times swam at the 2016 Rio Olympic games. Men (47.58) \& Women (52.70)
\item  Based on your linear model, during which year will the women be swimming faster than the men?
\end{enumerate}
%\customfoot

\end{document}

